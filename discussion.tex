% !TEX root = main.tex
\section{Future Work}
\label{sec:discussion-future-work}

\subsection{Analysis of Results}
Our immediate future work will involve analyzing our results of the survey and interviews. Open-ended responses will be coded to detect underlying values associated with responses, and quantitative analysis along with Value Dams and Flows (as described previously) will help instruct us as to which designs should be incorporated into StopInfo or avoided altogether.

Additionally, we also wish to triangulate our sources of data by drawing upon our system logs. While many of our logs are anonymous, we made sure to create a unique id that would allow us to evaluate the behavior of people contributing to the system. If the user signs in, we can associate their display name and/or e-mail address with their usage patterns as well, and determine if making their profile public had any effect on their behavior. As self-reports tend to inflate socially-desirable behavior (such as altruism) and reduce reports of undesirable behaviors (such as recognition or self-gain), we can use logs to help investigate the latter values. For example, is there a correlation between when a user makes their profile public and the amount of submissions they make? Did contributors provide more submissions during the period our group provided free bus tickets? Was the accuracy of these submissions lower, higher, or about the same as others provided anonymously? In other words: do contributors really value reputation, rewards, or recognition more than they say they do?

\pagebreak 

\subsection{Further Investigations}
As Value Sensitive Design encourages an iterative process that touches every phase of technology design, we will use the results of these second-round conceptual and empirical value investigations to iterate on our design and perform further conceptual, technical, and/or empirical investigations. One obvious investigation to pursue next would be a stakeholder analysis of blind and low vision or general users, as there may be new value tensions identified between them and StopInfo contributors. For example, early contributor survey results have already shown a large desire to know if a request is coming from a person who is blind or low vision (33\% of all respondents marked this as ``Very important`` to know, while 22\% marked ``Very important`` for any transit rider). Preliminary interview results have confirmed this finding, mentioning that they felt the information is more useful for someone who is blind or low vision, and thus they are more likely to respond to the request \emph{and} provide more detailed information. However, someone who is blind or low vision may not want to be singled out for their disability, or have contributors ``feel sorry'' for them. (It is worth noting that one of the contributor interviewees was blind and mentioned that she is okay with them feeling sorry for her, since the request is \emph{``more likely to be answered''}). 

\section{Conclusion}
In summary, we performed value-sensitive conceptual and empirical investigations with our StopInfo contributor stakeholder group in order to understand these individuals motives and determine the values implicated by our current and potential designs of StopInfo. Our findings will hopefully support future design decisions that can bolster participation as well as prevent potential misuses of the technology (e.g., entering false information or spam). Furthermore, once analysis of results are complete, we can help lend insights that extend to other geographically-situated contribution systems like StopInfo, such as Waze, OpenStreetMap, or MiFlight, and how contributions to these type of systems might be sustained over time.  