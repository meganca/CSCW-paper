% !TEX root = main.tex
\section{Related Work}
\label{sec:related-work}
As part of the conceptual investigation for uncovering motives and values surrounding information contributors' use of StopInfo, we first explored existing literature to try and identify potential values of interest for our own work.

\subsection{Motives for Contributing}
Much work has been done on investigating motives for contributions to crowdsourcing technologies such as Wikipedia \cite{oreg-2008, schroer-2009}, citizen science projects (e.g. Stardust@home, FoldIt, GalaxyZoo) \cite{eveleigh-2014, nov-2011, prestopnik-2011, reed-2013}, question and answer forums (e.g. Yahoo! Answers \cite{dearman-2010}, Math Overflow \cite{tausczik-2012}), open source projects \cite{oreg-2008, wu-2007}, and other knowledge-sharing systems \cite{miller-2007}. Additionally, Moore et al. investigated how type and purpose of a virtual community (e.g. wikis, blogs, or Internet forums) correlates with members' motives for participating \cite{moore-2007}. 

A system that provides detailed transit stop information can be
viewed as a specialized geowiki.  Literature on geowikis includes 
OpenStreetMap\footnote{http://openstreetmap.org}, which
is certainly the largest geowiki. Haklay
\cite{haklay-environment-planning-b-2010} provides an assessment of
the success of OpenStreetMap, both in terms of accuracy and coverage,
in comparison with Ordnance Survey datasets in the United 
Kingdom. Another notable geowiki is Cyclopath for bicyclists
\cite{panciera-chi-2010,priedhorsky-wikisym-2007},
which also includes route-finding capabilities. However, it is unclear whether motivations for contributing to wikis such as Wikipedia (which is perhaps the most well-studied) extend to geowikis. For example, Moore et al. identify the motives of \emph{altruism}, \emph{belonging}, \emph{collaboration}, \emph{egotism}, \emph{knowledge}, \emph{power}, \emph{reciprocity}, \emph{reputation}, and \emph{self-esteem} as pertaining to wiki participation \cite{moore-2007}. However, a system such as StopInfo or OpenStreetMap requires simple contextual knowledge (i.e., physical features that are present in a specific area) rather than expert knowledge; thus, it is unclear whether motives such as knowledge (in the sense of improving knowledge rather than sharing) or power might apply. 

Based on our present knowledge, contribution behavior and motivations surrounding crowdsourced transit information systems such as Waze or MiFlight are not currently well-studied. Work by Lathia et al. regarding transit riders' experiences with a crowdsourced transit information system for the London Underground called TubeStar mentions studying contributors' motives as future work \cite{lathia-2014}; however, to present, there is little work in this arena. 

\subsection{Reputation and Badge Systems}

Based on prior literature \cite{antin-2011, depaoli-2012, mamykina-2011}, we decided to implement a reputation system as well as a badge system as part of StopInfo. These works have shown that certain game elements, such as earning points and unlocking achievements (i.e., badges), have led to increased participation and have helped sustain contributions over time \cite{iacovides-2013}. Badges can also serve as a representation of \emph{community membership}, \emph{authority}, \emph{competence}, \emph{experience}, \emph{identity}, and \emph{reputation} \cite{halavais-2012}.
