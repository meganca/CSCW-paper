% !TEX root = main.tex
\section{Value Sensitive Design}
\label{sec:vsd}

Much of our research is ultimately motivated by attempting to better
support certain human values such as independence, safety, equity,
participation, respect, and community.  To approach these value
questions, we employ Value Sensitive Design
\cite{friedman-amis-2006}, a principled, systematic approach to the
consideration of human values in the design of information technology. The primary features of Value Sensitive Design are: consideration of
both direct and indirect stakeholders (that is, the users of
technology and those affected by the technology even though they do
not use it); a tripartite methodology, consisting of conceptual,
empirical, and technical investigations, iteratively and integratively
applied; and an interactional theory to the value implications of
technology.

In other prior Value Sensitive
Design work \cite{borning-ecscw-2005,borning-chi-2012}, the
researchers found it valuable to draw a distinction among stakeholder
values, explicitly supported values, and designer values --- an
important designer value for us is avoiding paternalism toward people
with disabilities.

In the work reported here, we focus on one key set of direct
stakeholders: the information contributors to the StopInfo application. Prior work has investigated values associated with other stakeholders of this application, notably blind and low vision users of StopInfo \cite{campbell-2014}. We also identified general users of StopInfo as direct stakeholders, and bus/train operators, orientation and mobility (O\&M) instructors, King County Metro representatives, and family and friends of blind and low vision users of StopInfo as key indirect stakeholders.

\subsection{Value Scenarios}
Other works utilizing Value Sensitive Design include \emph{value scenarios}, a technique for envisioning the effects of technology designs while in the formative stages of design and value priorities are still unknown \cite{nathan-2007}. Value scenarios are brief fictional descriptions meant to evoke social and value implications associated with a hypothetical technology or feature design. For example, Czeskis et al. use value scenarios to explore the potential impact of a hypothetical mobile location monitoring application on the relationship between parents and their teens \cite{czeskis-2010}.

\pagebreak

\subsection{Value Dams and Flows}
Another methodology introduced in previous work \cite{miller-2007} that we utilize here are Value Dams and Flows. \emph{Value Flows} are defined to be the technical features or organizational policies that a large percentage of stakeholders would like to see included as part of the overall system. Conversely, \emph{Value Dams} refer to the technical features or organizational policies that are strongly opposed, even by a small percentage of stakeholders. Taken together, dams and flows support the desires of the majority and can help alleviate the concerns of minority groups. 

